\documentclass[a4paper, 10pt]{scrartcl}

\usepackage[latin1]{inputenc}
\usepackage{ngerman}

\begin{document}

\tableofcontents
\newpage

\section{Einige Begriffe}

\subsection{Cepheiden}
Sterne, bei denen die Helligkeit streng periodisch schwankt.\\
Standardkerzen zur Entfernungsmessung.

\subsection{Harvard-Standard}
Klassifizierung der Sterne nach dem Aussehen ihres Lichtspektrums.

\subsection{Spektrallinien}
\subsection{Absorption}
\subsection{Emission}
\subsection{Kosmologische Konstante}
\subsection{Periheldrehung des Merkus}
\subsection{Shapiro-Effekt}
\subsection{Retrograde Bahn des Mars}
\subsection{Kosmische Hintergrundstrahlung}
\subsection{Unsch�rferelation}
\subsection{Raumzeit}
\subsection{Mindestquantum}
\subsection{Plancksche Wirkungsquantum}
\subsection{Teilchen-Antiteilchenpaar}
\subsection{Quantenfluktuationen}
\subsection{Heisenbergsche Unsch�rferelation}
\subsection{Mindestunsch�rfe}
\subsection{Vakuumpolarisation}
\subsection{Energiedichte und Druck}
\subsection{Quantenmechanisches Vakuum}
\subsection{Antigravitative Energiefreisetzung}
\subsection{M�glichkeitsdruck}
\subsection{Casimir-Effekt}
\subsection{Planckwelt}
\subsection{Unbestimmtheitsrelation}
\subsection{Planckmasse}
\subsection{Plancktemperatur}
\subsection{Planckzeit}
\subsection{Gluonen}
\subsection{Quarks}
\subsection{Higgsfeld}
\subsection{Impulserhaltung}
\subsection{Spin}
\subsection{Bosonen}
\subsection{Fermionen}
\subsection{Fermidruck}
\subsection{Pauliprinzip}
\subsection{Entropie}
\subsection{Phasen�bergang}
\subsection{Wasserstoff-Br�ckenbindung}
\subsection{Symmetriebruch}
\subsection{Verdampfungsw�rme}
\subsection{Kristallisationsw�rme}
\subsection{Unterk�hltes Wasser}
\subsection{Skalarfeld}
\subsection{Harmonischer Oszillator}
\subsection{Potential}
\subsection{Inflationsfeld}
\subsection{GUT-Kraft}
\subsection{Infinite monkey theorem}
\subsection{Falsches Vakuum}
\subsection{Kosmische Inflation}
\subsection{Ruhemasse}
\subsection{Curie-Temperatur}
\subsection{Thermisches Gleichgewicht}
\subsection{Inverse Comptonstrahlung}
\subsection{Primordiale Nukleosynthese}
\subsection{Deuteron, Helium, Lithium, Beryllium}
\subsection{Anti-Neutrino}
\subsection{Beta-Zerfall}
\subsection{Gebundene Neutronen}
\subsection{Rekombination}
\subsection{380000 Jahre}
\subsection{4000 K Grenze}
\subsection{Pechtropfen-Experiment}

\end{document}
